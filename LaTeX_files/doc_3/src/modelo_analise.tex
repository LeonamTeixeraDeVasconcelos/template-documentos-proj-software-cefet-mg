\section{Modelo de análise - Projeto de Dados}

   \noindent \cbox{
         \textbf{DER} - Deve existir em todos os projetos que envolvem banco de dados
   }
   
    \noindent \cbox{
         Anexar os DER gerados pelo ambiente de desenvolvimento.
   }
   
   \subsection{Dicionário de dados completo}
   
   {\color{red}{Exemplos:}}
   
   \begin{center}
       
        \begin{longtable}{|p{2cm}|p{2cm}|p{1cm}|p{3.5cm}|p{1.8cm}|p{1.5cm}|p{2cm}|}
       
            \hline
            
            \multicolumn{7}{|c|}{
                \textbf{ENTIDADE: {\color{blue}{CIDADE}}}
            } \endhead \hline
            
            \textbf{Nome} & 
            \textbf{Tipo} &
            \textbf{Tam}  &
            \textbf{Descrição} &
            \textbf{Máscara} &
            \textbf{Regra} &
            \textbf{Valores válidos} \\ \hline
            
            CdCidade &
            INTEGER &
            5 &
            Código da Cidade &
            99999 &
            $> 0$ & \\ \hline
            
            NmCidade &
            VARCHAR &
            40 &
            Nome da Cidade &
            @! &
            not null & \\ \hline
            
            SgEstado &
            CHAR &
            2 &
            Sigla do estado &
            @! &
            not null & \\ \hline
       
       \end{longtable}
       
       \captionof{table}{Tabela de detalhamento da entidade cidade. Fonte: autores.}
       
       \begin{longtable}{|p{2cm}|p{2cm}|p{1cm}|p{3.5cm}|p{1.8cm}|p{1.5cm}|p{2cm}|}
       
            \hline
            
            \multicolumn{7}{|c|}{
                \textbf{ENTIDADE: {\color{blue}{BAIRRO}}}
            } \endhead \hline
            
            \textbf{Nome} & 
            \textbf{Tipo} &
            \textbf{Tam}  &
            \textbf{Descrição} &
            \textbf{Máscara} &
            \textbf{Regra} &
            \textbf{Valores válidos} \\ \hline
            
            CdBairro &
            INTEGER &
            5 &
            Código do Bairro &
            99999 &
            $> 0$ & \\ \hline
            
            NmBairro &
            VARCHAR &
            40 &
            Nome do Bairro &
            @! &
            not null & \\ \hline

       \end{longtable}
       
       \captionof{table}{Tabela de detalhamento da entidade bairro. Fonte: autores.}
       
   \end{center}
   
   \noindent \cbox{
        Faça um dicionário completo
   }